\chapter{Methods}
\label{chapter2}

%Everything that comes under the `Methods' criterion in the mark scheme should be described in one, or possibly more than one, chapter(s).

\section{Justification of software choices}
** Discuss Jupyter Notebook as a learning tool and why it makes for a good solution with respect to the background research performed. Talk about: why Python is good for numerical computation and good for people with a mathematical background; why DOLFINx is a great package for implementing the Finite Element Method and what dependencies come with that; how version control is used and the benefits of good version control management **

\section{Justification of design choices}
** Talk about content plan, why I structured it the way I did and why I chose the content I did, always referring back to relevant research. Motivate the need for a mathematical understanding of the problem and it's solution instead of simply solving using packages you don't understand. This section should talk more about the education and mathematics, and should also touch upon the background of my target audience.**

\section{Agile project management description}
** Introduce Agile development as a concept and then discuss the details how how it was used. Benefits and drawbacks, as well as the importance of user feedback. Could talk about how user feedback was collected here but could also save that for the results and discussion section. Might be good to have it here though. **
