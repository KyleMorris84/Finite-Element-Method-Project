\chapter{Introduction and Background Research}

% You can cite chapters by using '\ref{chapter1}', where the label must
% match that given in the 'label' command, as on the next line.
\label{chapter1}

% Sections and sub-sections can be declared using \section and \subsection.
% There is also a \subsubsection, but consider carefully if you really need
% so many layers of section structure.
\section{Introduction}

%<A brief introduction suitable for a non-specialist, {\em i.e.} without using technical terms or jargon, as far as possible. This may be similar/the same as that in the 'Outline and Plan' document. The remainder of this chapter will normally cover everything to be assessed under the `Background Research` criterion in the mark scheme.>

%{\em** Start by explaining the problem. This includes a brief description of PDEs, the need for numerical solutions to such equations and general motivation for the subject. Then go on to talk about the need for education, future aims for the project (such as being turned into a module or maybe a free online resource), and the importance of user feedback.**}

Partial differential equations (PDEs) are some of the most important types of formulae in mathematics. They arise in every field of mathematically inclined science, such as physics, engineering, chemistry and more. PDEs are used to describe physical systems. Some PDEs describe dynamical systems; that is, systems that change with time. Others describe static systems, that only vary in space, or sometimes other quantities. This might include the movement of fluids, heat and waves, or the deformation of certain structures acting under a force. They are used to model population dynamics, chemical reactions, electromagnetic fields and are also central to quantum mechanics. PDEs are fundamental to our understanding of science and the world\cite{pde-introduction}.

Since PDEs are so important, the natural question is, how are they solved? Solving a PDE simply means obtaining an equation that describes the key variable explicitly, e.g. velocity of fluid particles given their coordinates in space, number of individuals in a population at any time, or amount of heat at a certain point on a metal rod. There are methods for solving certain types of PDE analytically, but generally PDEs are either incredibly difficult to solve, or in many cases, impossible.

Fortunately, methods have been developed to provide approximate solutions to PDEs, called numerical methods. These methods always have some degree of error, but by increasing the amount of computation performed, this error can be reduced. Numerical methods can involve an enormous amount of computation, making them the ideal candidates for modern computing, where processors perform billions of operations per second.

One such method is called the Finite Element Method (FEM), where the domain being solved over is discretised, and linear algebra techniques are used to construct a linear system of equations that can be numerically solved to obtain our approximate solution\cite{brenner-scott-fem}. The method has a strong mathematical foundation and it can be shown that the approximation does converge as the number of discrete elements increases\cite{bengzon-larson-fem}. The method is fantastically versatile, as it can be applied to any PDE, whether linear or non-linear, the only problem is that it requires a good mathematical understanding to implement.

The aim of this project is to address this issue and create a set of learning resources that can be used to bridge the gap in understanding required to use FEM. The goal is to guide students through the mathematics as well as instruct them on how to implement the method themselves. In this report I will describe the background research done to influence my design decisions, detail the writing process, and discuss the results obtained from user feedback.


% Must provide evidence of a literature review. Use sections
% and subsections as they make sense for your project.
% \section{Literature review}

%<This section heading is purely a suggestion -- you should subdivide this chapter in whatever manner you think makes most sense for your project. It may also make sense to spread the `Background Research' over more than one chapter, in which case they should be named sensibly.>

\pagebreak

\section{The Finite Element Method}

\subsection{How the Finite Element Method Works} \label{subsection:1.2.1}

As previously mentioned, FEM is a numerical method for solving differential equations. Most often, it is applied to partial differential equations. The general form of a PDE is shown in Equation \ref{eq:1}. Here, the variables are denoted $t, x_1, \cdots, x_n$, of which there are $n+1$. The order of the equation is the order of the highest derivative of $u$, denoted $m$. $u$ is the unknown function that being solved for. Note that $t$ may not always be present here.
\begin{align}\label{eq:1}
F \, \Big(u, t, x_1, x_2, &\cdots, x_n \nonumber \\
\frac{\partial u}{\partial x_1}, \frac{\partial u}{\partial x_2}, &\cdots, \frac{\partial u}{\partial x_n}, \nonumber \\ 
\frac{\partial^2 u}{\partial^2 x^2_1}, \frac{\partial^2 u}{\partial x^2_2}, &\cdots, \frac{\partial^2 u}{\partial x_n^2}, \\
&\cdots \nonumber \\
\frac{\partial^m u}{\partial x^m_1}, \frac{\partial^m u}{\partial x^m_2}, &\cdots, \frac{\partial^m u}{\partial x_n^m} \Big)= 0\nonumber
\end{align}

Clearly, Equation \ref{eq:1} is a complicated equation, but most PDEs do not have this many variables. Also, it is rare that a PDE will have an order greater than two, although this can occur. Along side the equation itself, there will also be a set of conditions that constrain the variables to some domain $\Omega$. There will also be a set of conditions that constrain $u$ on the boundary of this domain $\delta \Omega$. These are called the boundary conditions. If one of the variables involved is time this must also be constrained by an initial condition. Together these make a initial value boundary problem (IVBP). A full instance can be seen in Equation \ref{eq:2}.

\begin{align}
\begin{cases} 
F=0 \quad &\text{on } \Omega \\
u = u_D(x_1, \cdots, x_n, t) \quad &\text{on } \delta\Omega \label{eq:2} \\
u = u_{\text{init}}(x_1, \cdots, x_n) \quad &\text{at } t=0
\end{cases}
\end{align}

$$
\text{where } \Omega = [0, 1] \times [0, 1]
$$

The method of solving such an equation using the finite element method first starts by converting Equation \ref{eq:2} into a variational formulation. Specific methods of constructing this formulation vary from problem to problem but the general approach is as follows:

\begin{itemize}
    \item Take the equation defined over $\Omega$ and multiply both sides by a test function $v$.
    \item Integrate both sides over $\Omega$.
    \item Integrate any second order derivatives by parts.
    \item Rearrange so all terms involving $u$ are on one side. This expression is called the bilinear form, denoted $a(u,v)$ and the other side is called the linear form, denoted $L(v)$.
\end{itemize}

After doing this, a few facts about the functions spaces $u$ and $v$ belong to need to be acknowledged. The class of functions that can solve Equation \ref{eq:2} belong to the function space,
$$V=\{ v : v \in H^1(\Omega), v=u_D \text{ on } \delta\Omega, v=u_\text{init} \text{ at } t=0 \}$$
Clearly, in addition to this, they must also satisfy the main equation. $H_1(\Omega)$ is the Sobloev space, which is a space where all functions must be square integrable and continuous and their derivatives must simply be square integrable. $V$ is called the trial space, and $u$ is called the trial function. In order for the variational formulation to be valid, the test function must also come from $H_1(\Omega)$, but instead include the condition that it must vanish on the boundary of our domain. This yields the test space,
$$V_0 = \{ v : v \in H^1(\Omega), v=0 \text{ on } \delta\Omega, \}$$

This trick allows us to remove any parts of our integration that include $v$ on $\delta\Omega$. More importantly however, subspaces of $V$ and $V_0$, denoted $V_h$ and $V_{h,0}$ can be constructed, where the elements are piecewise polynomial functions. This can only be done because these spaces permit functions with discontinuous derivatives. So the transition from a continuous space to a discontinuous one is made, replacing $u$ with $u_h \in V_0$ and assuming $v \in V_{h,0}$. This is the finite element approximated problem.
\begin{align}
&\text{Solve for $u_h \in V_h$ where,} \nonumber \\
&a(u_h,v) = L(v) \quad \forall v \in V_{h,0} \label{eq:3}
\end{align}

To solve this problem, our domain $\Omega$ must be split into discrete segments, called a mesh. In one dimension, this is amounts to splitting the number line into sub-intervals, in two dimensions, the plane is split into cells, or finite elements. These cells are primitive shapes like rectangles or triangles and depending on the domain, either it can be simpler to use one or the other. The points where these cells meet each other are called nodes.

Now to solve, techniques from linear algebra are used. A basis of $V_h$ and $V_{h,0}$ must be constructed to rewrite $u$ and $v$ as a linear combination of basis functions. The one used in the finite element method is called the nodal basis, because the coefficients of a linear combination of each of these basis vectors are equal to the function at the nodes of the mesh. The usefulness of this, is that any function can be defined using only these nodal values, meaning if they can be obtained, the approximate solution can be found.

The way to obtain these coefficients, called degrees of freedom, is by substituting in $u$ and $v$ as a linear combination of these basis functions, called hat functions. This produces $n+1$ independent equations, where $n$ is the number of nodes in the mesh. Then, the resulting system of equations can be solved using known numerical methods. One can use direct methods like Gaussian elimination and LU-Factorisation, or where appropriate, iterative methods like Jacobi, or Gauss-Seidel.

If time is a variable in the equation, one can use the numerical methods used for ordinary differential equations to discretise the time domain and obtain an approximation that way. The Runge-Kutta methods are a popular choice.

\subsection{Convergence and Error Estimates}

FEM is an extremely powerful technique because it can be applied to any problem of the form \ref{eq:1} if a variational form can be defined, and it can be shown that all problems have such a formulation \cite{e-toni}. There may be some practical difficulties along the way, but the sophistication of simultaneous equation solvers means that the most computationally intense problems can be scaled to work on high-performance clusters.

One natural question that arises is that of convergence. Convergence of the method can be proved by showing that the error of our approximated solution tends to $0$ as the mesh size increases. A natural way to measure error is using the $L^2$-norm. This is because the Sobloev space $H^1$ is actually a Hilbert space, defined with the $L^2$-norm as its inner product. Thus, the $L^2$-norm is defined for all elements of $V_h$. The $L^2$ inner product of two functions $f$ and $g$ is defined as 
$$
\langle f, g \rangle_{L^2(\Omega)} = \int_{\Omega} f \bar{g} \, \mathrm{d}x
$$
Where $\bar{g}$ is the complex conjugate of $g$. Then, the $L^2$-norm is the square root of the inner product of a function with itself, denoted

$$
\lVert v \rVert_{L^2(\Omega)} = \sqrt{\int_{\Omega} \lvert v^2 \rvert \, \mathrm{d}x} 
$$

An enlightening example would be $\mathbb{R}^n$, which is also a Hilbert space when equipped with the familiar "dot" product as its inner product. The norm can then be though of as the distance between two points if the following is calculated,

$$\lVert x - y \rVert = \sqrt{(x-y) \cdot (x-y)} = \sqrt{\sum_{i=0}^n (x_i-y_i)^2}$$

This is precisely the definition of distance between two points. $H^1$ works the same way, only with a different notion of "distance", the $L^2$-norm $\Vert u - v \Vert_{L^2(\Omega)}$. Then, the error of our approximation $u_h$ from the actual solution $u$ can be measured by $\Vert u - u_h \Vert_{L^2(\Omega)}$.
In Bengzon and Larson's textbook on the finite element method\cite{bengzon-larson-fem}, they find the best estimate of error to be 
\begin{align}
\Vert u - u_h \Vert_{L^2(\Omega)} \leq Ch^2\Vert D^2 u \Vert_{L^2(\Omega)} \label{eq:4}
\end{align}

Where $C$ is a constant, $D^2u$ is the second order total derivative of $u$ and $h$ is defined to be the longest edge of any cell in the mesh, which is just a way of measuring the mesh size. It is clear from \ref{eq:4} that as $h \rightarrow 0$, the error also tends to $0$, which means the finite element converges as the size of the cells in the mesh decreases.

\subsection{Overview of Tools}

The process outlined in Section \ref{subsection:1.2.1} is difficult to implement without any libraries, even for the simplest of problems. This is due to the mathematical complexity of all the operations that would need to be performed. The functions one would need to implement include numerical integration, evaluating hat functions, partial derivatives, linear algebra solvers, and much more. This is why, most people prefer to use tools in order to perform finite element analysis. These tools come in the form of libraries and modules, to bespoke programs. Each have their benefits and advantages, and we will analyse these now.

\section{Overview of Literature Regarding Mathematical Education}

** Discuss different taxonomies, best practices, and approaches. Reference empirical studies if possible. **

\subsection{Educational Practices}

In order to write the learning resources effectively, a concrete understanding of the Finite Element Method is needed. However an equally, and often forgotten part, of writing a good learning resource involves having a good knowledge of education and the utilisation of an approach that puts the student experience first.

``How to write a good textbook / learning resource - blooms taxonomy, learning objects''

\subsection{Types of Learning resource}

``Textbooks, in-person courses, video, blended learning, e-learning and mobile learning in mathematics education''

\subsection{Application to Mathematics}

``Khan academy, brilliant, codecademy''

\section{Overview of Existing FEM Learning Resources}

** Review a few online tutorials, discuss what is good and bad about each with respect to the previous subsection. Do the same for books. Also a good point to review some of the materials I want to base the notebooks off of and explain why I chose them. **

\subsection{Book by Brenner-Scott}

\subsection{Book by Bengzon-Larsen}

\subsection{Tutorial by Jorgen S Dokken}