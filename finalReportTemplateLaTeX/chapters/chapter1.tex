\chapter{Introduction and Background Research}

% You can cite chapters by using '\ref{chapter1}', where the label must
% match that given in the 'label' command, as on the next line.
\label{chapter1}

% Sections and sub-sections can be declared using \section and \subsection.
% There is also a \subsubsection, but consider carefully if you really need
% so many layers of section structure.
\section{Introduction}

%<A brief introduction suitable for a non-specialist, {\em i.e.} without using technical terms or jargon, as far as possible. This may be similar/the same as that in the 'Outline and Plan' document. The remainder of this chapter will normally cover everything to be assessed under the `Background Research` criterion in the mark scheme.>

** Start by explaining the problem. This includes a brief description of PDEs, the need for numerical solutions to such equations and general motivaiton for the subject. Then go on to talk about the need for education, future aims for the project (such as being turned into a module or maybe a free online resource), and the importance of user feedback. Here is a citation \cite{parikh1980adaptive}. **


% Must provide evidence of a literature review. Use sections
% and subsections as they make sense for your project.
% \section{Literature review}

%<This section heading is purely a suggestion -- you should subdivide this chapter in whatever manner you think makes most sense for your project. It may also make sense to spread the `Background Research' over more than one chapter, in which case they should be named sensibly.>

\section{Overview of literature regarding mathematical education}

** Discuss different taxonomies, best practices, and approaches. Reference empirical studies if possible. **

\begin{itemize}
	\item Bloom's Taxonomy!
	\item Effective Mathematical Teaching Principles
	\item Introduce the idea of Learning Objects
\end{itemize}



\section{Analysis of available learning resources on FEA}

** Review a few online tutorials, discuss what is good and bad about each with respect to the previous subsection. Do the same for books. Also a good point to review some of the materials I want to base the notebooks off of and explain why I chose them. **
